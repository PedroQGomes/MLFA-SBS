\documentclass[a4paper,10pt]{article}
\usepackage[utf8]{inputenc}

%opening
\title{Relatório TP1 SBS-MLFA}
\author{Luis Freitas; Daniel Pereira}

\begin{document}

\maketitle

\begin{abstract}

\end{abstract}

\section{Introdução}
\newpage
\section{Metodologias}
\subsection{CRISP-DM}
\subsection{Modelo Baseados em Árvores}
\newpage



\section{Arquiteturas e Ferramentas}
\newpage

\section{1º Dataset}

\newpage


\subsection{Contexto do Dataset}
%escrever por palavras nossas
“O training set deverá ser usado para desenvolver e treinar o modelo de Machine Learning. No training set é fornecida informação referente ao nível de incidentes rodoviários (accidents) de cada registo para que possam treinar os modelos de aprendizagem. O modelo a desenvolver deverá ter, na sua base, features como a magnitude do atraso que se verifica numa determinada hora, o tempo de atraso provocado pelos incidentes, a temperatura, pressão atmosférica e a velocidade do vento, entre outras features que caracterizam um determinado ponto temporal.”

“O test set será utilizado para validar a performance do modelo em dados ainda não vistos pelo mesmo. Neste dataset de teste não é fornecida a class referente ao nível de incidentes rodoviários. O vosso trabalho é prever essa mesma class. Devem utilizar o modelo desenvolvido para prever, para cada registo do test set, o nível de incidentes rodoviários esperados.”

É também disponibilizado um exemplo de um ficheiro de submissão (example_submission.csv) onde é assumido que o nível de incidentes rodoviários esperado para cada registo é Medium. 
\subsection{Análise e Compreensão dos Dados}
\subsection{Modelação}
\subsection{Análise de Resultados}
\section{Dataset de Competição}


\newpage
\subsection{Contexto do Dataset}
\subsection{Análise e Compreensão dos Dados}
\subsection{Modelação}
\subsection{Análise de Resultados}
\section{Conclusão}

\end{document}
