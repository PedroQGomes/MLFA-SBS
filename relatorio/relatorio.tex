\documentclass[a4paper,10pt]{article}
\usepackage[utf8]{inputenc}

%opening
\title{Relatório TP1 SBS-MLFA}
\author{Luis Freitas; Daniel Pereira}

\begin{document}

\maketitle

\begin{abstract}

\end{abstract}

\section{Introdução}
\newpage
\section{Metodologias}
\subsection{CRISP-DM}
\subsection{Modelo Baseados em Árvores}
\newpage



\section{Arquiteturas e Ferramentas}
\newpage

\section{Data Set da Competição}

\newpage


\subsection{Contexto do Dataset}
%escrever por palavras nossas
O dataset utilizado neste projeto é referente ao nível de incidentes rodoviários na zona de Braga e contém algumas features que representam a magnitude do atraso que se verifica a cada hora, a temperatura, a pressão atmosférica, a velocidade do vento e ainda outras features. 

O principal objetivo é aprimorar um modelo de Machine Learning Baseado em Arvores mas também fazer uma analise exploratoria completa dos dados para se retirar algumas informações importantes do dataset. Outo ponto passa por conseguir preprarar o dataset com técnicas de engenharia de dados para se obter os melhores resultados possiveis. Este dataset é referente a uma competição da plataforma Kaggle

Em relação as features do dataset estas são apresentadas da seguinte forma:

1)city_name - nome da cidade em causa;
2)record_date - o timestamp associado ao registo;
3)magnitude_of_delay - magnitude do atraso provocado pelos incidentes que se verificam no record_date correspondente;
4)delay_in_seconds - atraso, em segundos, provocado pelos incidentes que se verificam no record_date correspondente;
5)affected_roads - estradas afectadas pelos incidentes que se verificam no record_date correspondente;
6)luminosity - o nível de luminosidade que se verificava na cidade de Braga;
7)avg_temperature - valor médio da temperatura para o record_date na cidade de Braga;
8)avg_atm_pressure - valor médio da pressão atmosférica para o record_date na cidade de Braga;
9)avg_humidity - valor médio da humidade para o record_date na cidade de Braga;
10)avg_wind_speed - valor médio da velocidade do vento para o record_date na cidade de Braga;
11)avg_precipitation - valor médio de precipitação para o record_date na cidade de Braga;
12)avg_rain - avaliação qualitativa do nível de precipitação para o record_date na cidade de Braga;
13)accidents - indicação acerca do nível de incidentes rodoviários que se verificam no record_date correspondente na cidade de Braga.

 
\subsection{Análise e Transformação dos Dados}

\subsection{Modelação}
\subsection{Análise de Resultados}
\section{Dataset de Competição}


\newpage
\section{DataSet 1}
\subsection{Contexto do Dataset}
\subsection{Análise e Compreensão dos Dados}
\subsection{Modelação}
\subsection{Análise de Resultados}
\section{Conclusão}

\end{document}
